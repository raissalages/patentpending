O objetivo deste trabalho é desenvolver uma bengala inteligente direcionada para pessoas com deficiência visual, utilizando sensores e módulos para criar um sistema de detecção de objetos e fornecer alertas táteis e auditivos ao usuário. A proposta da bengala é servir como uma ferramenta de tecnologia assistiva, capacitando os usuários a interagir com o ambiente ao seu redor, promovendo a autonomia e estimulando a independência das pessoas afetadas pela perda total ou parcial da visão. Adicionalmente, destaca-se que este projeto tem como objetivo ser uma solução de baixo custo com um propósito social significativo. Ao buscar uma abordagem acessível e economicamente viável, busca-se democratizar o acesso à tecnologia assistiva para pessoas com deficiência visual, ampliando assim o impacto positivo desta iniciativa na sociedade. A combinação de uma tecnologia eficiente com um compromisso com a acessibilidade financeira reflete o comprometimento deste trabalho com a equidade e a inclusão, visando tornar a bengala inteligente uma ferramenta acessível para um amplo espectro de usuários.

% Separe as palavras-chave por ponto
\palavraschave{deficiência visual; tecnologia assitiva; bengala inteligente; sistemas embarcados.}