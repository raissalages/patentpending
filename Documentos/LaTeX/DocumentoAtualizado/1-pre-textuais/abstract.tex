The aim of this work is to develop a smart cane aimed at visually impaired individuals, using sensors and modules to create an object detection system and provide tactile and auditory alerts to the user. The proposal of the cane is to serve as an assistive technology tool, empowering users to interact with their surrounding environment, promoting autonomy, and stimulating independence for individuals affected by total or partial vision loss. Additionally, it is highlighted that this project aims to be a low-cost solution with significant social purpose. By seeking an accessible and economically viable approach, the goal is to democratize access to assistive technology for visually impaired individuals, thus expanding the positive impact of this initiative on society. The combination of efficient technology with a commitment to financial accessibility reflects the dedication of this work to equity and inclusion, aiming to make the smart cane an accessible tool for a wide spectrum of users.

% Separe as Keywords por ponto e vírgula.
\keywords{visual impairment; assistive technology; smart cane; embedded systems.}