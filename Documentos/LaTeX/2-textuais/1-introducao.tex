\chapter{Introdução}
\label{cap:introducao}

%Para começar a usar este \textit{template}, na plataforma \textit{ShareLatex}, vá nas opções (três barras vermelhas horizontais) no canto esquerdo superior da tela e clique em "Copiar Projeto" e dê um novo nome para o projeto. 

A Constituição Federal de 1988, documento produzido com grande enfoque na instituição dos direitos humanos após o processo de redemocratização no Brasil, reforça a ampliação do acesso a serviços e informações para pessoas com deficiência através do Artigo 24. No mesmo, foi definido que uma deficiência caracteriza-se por um impedimento de longo prazo em aspectos físicos, mentais, intelectuais e/ou sensoriais, podendo afetar a participação plena de um indivíduo na sociedade. No entanto, apesar dos avanços legislativos e do reconhecimento dos direitos das pessoas com deficiência, ainda existem desafios significativos no que diz respeito à efetiva implementação dessas medidas e à garantia de igualdade de oportunidades para todos os cidadãos.

Segundo o senso levantado pelo Instituto Brasileiro de Geografia e Estatística (IBGE) em 2022, a dificuldade para enxergar foi a segunda mais observada na população, atingindo mais de 6 milhões de pessoas. O IBGE considera uma pessoa com deficiência aquela que apresenta muita dificuldade ou total incapacidade de realizar atividades voltadas a certos domínios funcionais, como visão, locomoção, audição, cognição, comunicação e autocuidado   \cite{ibge}. Nesse sentido, uma deficiência visual é uma condição que afeta a visão de uma pessoa, total ou parcialmente, e que pode ter origem congênita ou ser adquirida ao longo da vida por meio de uma doença, lesão ou envelhecimento.

No entanto, a extensão da desigualdade transcende as barreiras físicas e sensoriais, estendendo-se de maneira preocupante para o âmbito econômico. Os dados revelam que o rendimento médio habitual de pessoas com deficiência é, em média, 31\% menor do que aquele apresentado por pessoas sem qualquer tipo de dificuldade mencionada \cite{ibge}. Esse contraste entre a presença de deficiências e o impacto econômico sublinha uma disparidade significativa que impõe barreiras substanciais às oportunidades de vida, educação, emprego e participação plena na sociedade. Não obstante, os dados auferidos destacam não apenas a presença significativa dessas pessoas na população total, mas também a urgência de um olhar mais específico e direcionado para a implementação de medidas que possam superar as dificuldades únicas enfrentadas pelas pessoas com deficiência visual.

A complexa interseção entre desigualdades sociais e econômicas se torna ainda mais evidente quando consideram-se as dificuldades enfrentadas pela comunidade de deficientes visuais no Brasil. Além dos desafios físicos e sensoriais diários, esses indivíduos também lutam para superar barreiras no acesso à informação, educação, cultura e oportunidades profissionais. Embora o avanço tecnológico tenha trazido algumas melhorias, ainda há uma lacuna significativa na adaptação de conteúdos para formatos acessíveis, como a escrita em Braille ou tecnologias de leitura de tela. Como resultado, muitas pessoas com deficiência visual encontram-se excluídas de atividades fundamentais para o desenvolvimento pessoal e profissional, restringindo seu pleno engajamento na sociedade e exacerbando as desigualdades existentes.

Um dos desafios mais complexos enfrentados pelas pessoas com deficiência visual é a inclusão social. A falta de conscientização na sociedade contribui diretamente para a exclusão e marginalização desses indivíduos, resultando em preconceitos que dificultam significativamente sua interação e participação plena nas atividades sociais. Além disso, essa falta de conscientização se estende ao ambiente de trabalho e ao acesso aos serviços de saúde, onde a escassez de vagas adaptadas e a discriminação no ambiente profissional limitam severamente sua autonomia e oportunidades de desenvolvimento pessoal e profissional. Apesar da Lei nº 8.213/91 estabelecer que o quadro de funcionários de uma empresa possua de 2\% a 5\% de funcionários portadores de deficiência, as empresas ainda encontram dificuldade no preenchimento dessas vagas por não encontrarem mão de obra qualificada e não possuírem estrutura para o recebimento dessas pessoas \cite{pcd-mercado-trabalho-2}. 

Por causa de todos esses desafios, as tecnologias assistivas para pessoas com deficiência visual estão evoluindo, marcada com avanços significativos que visam promover a autonomia e inclusão na sociedade. Nesse sentido, o presente trabalho propõe-se a desenvolver uma bengala inteligente, que conta com a presença de sensores ultrassônicos e transdutores, com o propósito de auxiliar na interação entre o usuário e o ambiente circundante.

Essa tecnologia se propõe a melhorar a qualidade de vida das pessoas com deficiência visual que precisam da assistência, tornando-as mais independentes, capazes de se locomover sem ajuda de terceiros e promovendo sua autonomia. Desse modo, se torna necessário soluções acessíveis e eficazes, que devem ser projetadas levando em consideração suas necessidades e limitações, barreiras criadas pela falta de acessibilidade, para satisfazer o usuário.


A bengala inteligente é projetada para permitir às pessoas com deficiência visual compreender e interagir melhor com o ambiente ao seu redor, podendo assim superar as limitações que enfrentam no cotidiano com seu uso. Com o intuito de priorizar requisitos de design que enfatizem a experiência do usuário, a equipe de desenvolvimento adotou uma abordagem centralizada no usuário.

Este documento fornece um guia completo com o objetivo de fornecer informações detalhadas sobre o design, desenvolvimento e implementação da bengala para desenvolvedores, designers e usuários finais, explorando diferentes componentes e tecnologias que foram utilizados na bengala inteligente, assim como também suas funções e aplicações práticas.

A bengala inteligente se esforça para preencher a lacuna nas soluções de mobilidade para pessoas com deficiência visual, fornecendo uma alternativa sofisticada e acessível às bengalas tradicionais. Além disso, pretende melhorar a experiência de mobilidade, bem como promover a sua autonomia e inclusão social. Incorporando recursos inteligentes como detecção de obstáculos e interação sensorial com o usuário, a bengala inteligente se destaca como uma ferramenta versátil e eficaz que atende às necessidades específicas desse público.

A importância das bengalas inteligentes vai além dos aspectos técnicos. Este é um avanço inovador em tecnologia assistiva que demonstra o potencial da inovação para transformar vidas e criar oportunidades. Combinando investigação científica, design centrado no usuário, desenvolvimento tecnológico e comprometimento com um baixo custo, ela visa proporcionar experiências inovadoras e inclusivas para pessoas com deficiência visual. Neste contexto, o documento procura apresentar o projeto de forma abrangente, desde o seu conceito até à sua aplicação prática e potencial impacto na sociedade. A seguir, discutimos os desafios enfrentados pelas pessoas com deficiência visual, as soluções oferecidas pela bengala inteligente, objetivos, justificativa, público-alvo, tecnologias, arquitetura, análise competitiva e outros aspectos relacionados ao desenvolvimento e implementação deste suporte tecnológico inovador.



\section{CONTEXTO}
As bengalas inteligentes surgiram como resposta a uma série de problemas que as pessoas com deficiência visual enfrentam nas suas atividades diárias. Esses desafios incluem barreiras físicas como móveis, escadas e obstáculos em ambientes urbanos que podem comprometer sua segurança e independência. Além disso, a falta de informações contextuais em tempo real sobre o ambiente ao seu redor pode limitar sua capacidade de navegar com confiança e eficácia.

Um grande problema centraliza-se na dificuldade em identificar objetos específicos ou pontos de referência importantes no percurso durante a locomoção, o que pode levar a situações desconfortáveis ou perigosas. A falta de  respostas imediatas acerca da presença de obstáculos pode ter um impacto negativo na mobilidade e na confiança destes indivíduos, afetando a sua qualidade de vida de maneira generalizada. Outro desafio é a disponibilidade limitada de informações relevantes, como placas, cardápios de restaurantes e placas de identificação em espaços públicos. A dependência excessiva de outras fontes de tais informações pode limitar a autonomia e a independência das pessoas com deficiência visual, resultando em experiências desiguais na sociedade.


O maior desafio é a lacuna entre a tecnologia existente e as necessidades reais dos utilizadores com deficiência visual. Soluções tradicionais, como bengalas e cães-guia, têm limitações no fornecimento de informações detalhadas e em tempo real sobre o ambiente circundante. É, portanto, importante colmatar esta lacuna através de abordagens inovadoras e tecnologicamente avançadas, como bengalas inteligentes que podem transformar a experiência de um indivíduo.


\section{PROBLEMA}
As pessoas com deficiência visual têm mobilidade limitada nos mais diversos ambientes. Devido à deficiência visual, essas pessoas têm dificuldade em identificar obstáculos no caminho, o que pode causar situações desconfortáveis ou perigosas. Por exemplo, lidam com obstáculos físicos, como móveis, degraus e obstáculos no ambiente urbano, como atravessar a rua e a falta de piso tátil, que alerta mudanças de direção e presença de algum obstáculo, pois não estão presentes em muitas calçadas. Muitos possuem dependência de outra pessoa ou recursos específicos devido à acessibilidade limitada a informações, como cardápios em restaurantes ou placas de identificação.

As ferramentas tradicionais, como bengalas convencionais e cães-guia têm suas próprias limitações que afetam a pessoa com deficiência visual. As bengalas convencionais não conseguem fornecer informações detalhadas sobre obstáculos no ambiente, enquanto cães-guia precisam de treinamento específico, nem todos os cães conseguem ser treinados e não são acessíveis a todos. Como exemplo disso, Silva e Duarte (2018) exploram a relação entre cães guias e a parcela da população que precisa desse auxílio, expondo que há mais de 10 mil pessoas com grandes difiucldades visuais na cidade de Caruaru - Pernambuco, entretanto, não existe um cachorro treinado para prestar esse tipo de serviço \cite{cao-guia-ta}. Dessa forma, evidencia-se a alta demanda que não pode ser suprida pelos motivos supracitados.

A acessibilidade inadequada em espaços públicos e edificações representa outro problema enfrentado pelos deficientes visuais. Eles precisam de diversas adaptações para receber essas pessoas, como rampas, sinalização tátil, leitores e outros, que tornariam mais fácil circular com segurança e independentemente em locais como ruas, praças, escolas, etc.

A dependência de terceiros ou ferramentas tradicionais representa mais um desafio. As pessoas com deficiência visual podem acabar dependendo de familiares, amigos ou profissionais de saúde para a realização de suas tarefas cotidianas, limitando sua autonomia e independência.

Limitações na navegação e acesso a informações visuais aumentam o risco de acidentes e situações adversas no dia a dia por causa do desconhecimento de obstáculos, sinais e informações visuais. A falta de segurança e independência na locomoção também limita a liberdade de ir e vir de maneira segura e independente.

Além disso, existem barreiras para participação na sociedade, impedindo esses indivíduos de frequentar espações públicos, instituições educacionais, locais de trabalho e eventos culturais devido à falta de acessibilidade.


A necessidade de soluções mais eficazes e inovadoras é cada vez mais urgente. Existe uma demanda crescente por tecnologias acessíveis que possam enfrentar os desafios dessa comunidade de maneira eficaz. Esse reconhecimento reflete uma crescente valorização da acessibilidade e da inclusão no desenvolvimento de produtos e serviços tecnológicos.

\section{SOLUÇÃO}
A solução proposta para os problemas enfrentados pelas pessoas portadoras de deficiências visuais é a bengala inteligente, um dispositivo inovador que combina tecnologia de sistemas embarcados e design ergonômico para fornecer informações em tempo real. A bengala está equipada com sensores e dispositivos de rastreamento que podem identificar obstáculos, pontos de referência importantes e outras informações relevantes no ambiente circundante.

A bengala inteligente usa tecnologias como sensores e outros dispositivos para coletar dados sobre seu ambiente e transmitir essas informações ao usuário por meio de feedback, de áudio ou tátil. Por exemplo, ao se aproximar de um obstáculo, o usuário pode receber uma vibração ou notificação sonora indicando a presença e localização relativa do obstáculo. Isto permite aos utilizadores aceder a informações contextuais essenciais para a segurança e autonomia, mesmo em ambientes desconhecidos ou dinâmicos.
As soluções também abordam questões de acessibilidade, fornecendo uma interface intuitiva e personalizável que atende às necessidades individuais dos usuários. Controles simples permitem que os usuários interajam facilmente com o dispositivo e ajustem as configurações de acordo com suas preferências e necessidades.
Por outras palavras, as bengalas inteligentes são uma solução abrangente e eficaz que melhora a qualidade de vida e promove a independência das pessoas com problemas de visão. Combinando tecnologia inovadora e design centrado no usuário, estes dispositivos têm o potencial de transformar a experiência de navegação e interação em ambientes urbanos, promovendo uma sociedade mais inclusiva e acessível.

\section{OBJETIVOS}

A bengala inteligente visa fornecer soluções abrangentes e eficazes para melhorar a mobilidade, a segurança e a independência das pessoas com deficiência visual, utilizando tecnologia de ponta para desenvolver um dispositivo que identifique obstáculos, pontos de referência importantes e outras informações relevantes do ambiente ao redor, fornecendo feedback em tempo real aos usuários.

Os principais objetivos são aumentar a segurança dos usuários, fornecendo notificações e feedback táteis ou auditivos para ajudá-los a evitar obstáculos e proporcionar uma experiência de navegação mais segura e suave em uma variedade de ambientes; aprimorar a mobilidade, fornecendo informações contextuais sobre sua localização e arredores para auxiliar na orientação em ambientes internos e externos; promover a independência, ajudando os usuários a explorar novos espaços e a realizar as atividades diárias de forma mais autônoma, reduzindo a dependência de ajuda externa; e promover a acessibilidade, fornecendo interações simplificadas e personalizadas, integrando uma interface intuitiva e personalizável que atenda às necessidades individuais dos usuários.

A bengala inteligente pretende ser uma ferramenta versátil e eficaz para pessoas com deficiência visual, fornecendo informações importantes sobre o ambiente de forma acessível e intuitiva. Ao integrar tecnologias inovadoras e um design centrado no usuário, esses dispositivos têm o potencial de ter um impacto positivo na qualidade de vida dos indivíduos e na inclusão social.


\section{JUSTIFICATIVA}
A bengala inteligente é uma resposta inovadora e necessária aos desafios enfrentados pelas pessoas com deficiência visual, proporcionando uma solução prática e tecnologicamente avançada para aumentar a mobilidade e a independência. Em primeiro lugar, é importante aumentar a segurança e a autonomia dos portadores de deficiências visuais. Muitas pessoas enfrentam obstáculos todos os dias ao navegar em ambientes desconhecidos ou difíceis, o que pode afetar a sua confiança e limitar as suas atividades diárias. A bengala inteligente visa reduzir essas barreiras, fornecendo avisos e feedback úteis, permitindo que os usuários se movam com mais confiança e independência.

Além disso, a tecnologia avançada incorporada à bengala inteligente pode oferecer uma experiência personalizada. Ao integrar sensores e dispositivos de localização de alta precisão, os dispositivos podem fornecer informações contextuais precisas sobre o ambiente ao seu redor. Isto não só melhora a mobilidade, mas também permite que os usuários naveguem mais facilmente em novos espaços.
Outra razão importante para a criação de uma bengala inteligente é a importância da acessibilidade e da inclusão. Num mundo cada vez mais digital e tecnológico, é importante garantir que as pessoas com deficiência visual tenham igualdade de acesso a oportunidades e facilidades. Estes dispositivos podem ajudar a reduzir as barreiras à acessibilidade física e promover a participação ativa na sociedade. Também oferece uma abordagem inovadora para o desenvolvimento de tecnologia assistiva. Integrando hardware avançado com software inteligente e interfaces personalizáveis, esses dispositivos são um passo importante na criação de soluções mais eficientes e centradas no usuário.

Finalmente, as bengalas inteligentes têm o potencial de impactar positivamente a qualidade de vida das pessoas com deficiência visual, promovendo maior independência, segurança e inclusão social. Ao abordar os desafios reais que estas comunidades enfrentam, este projeto demonstra o papel transformador da tecnologia na promoção da igualdade de oportunidades e na melhoria da qualidade de vida para todos.

\section{PÚBLICO ALVO}
A cada cinco segundos, uma pessoa em todo o mundo fica com deficiência visual. A pesquisa vem do projeto da Organização Mundial da Saúde (OMS), Relatório Mundial sobre Deficiência 2010 e Visão 2020 (um plano para acabar com a cegueira evitável até 2020). 90\% de todos os casos de cegueira ocorrem em países em desenvolvimento e subdesenvolvidos \cite{vision-2010}. 

Segundo dados do IBGE de 2010, mais de 6,5 milhões de pessoas no Brasil sofrem com problemas de visão \cite{mec-direitos-pcd}. Destes, 528.624 pessoas são cegas. 6.056.654 pessoas têm um problema grave permanente de visão (baixa visão ou visão subnormal). Outros 29 milhões de pessoas relataram ter problemas de visão persistentes mesmo quando usavam óculos ou lentes de contato. Em geral, 23,9\% (45,6 milhões de pessoas) da população total do Brasil relatam ter algum tipo de deficiência, sendo a mais comum a visão, afetando 3,5\% da população. Em seguida estão os problemas motores (2,3\%), intelectuais (1,4\%) e auditivos (1,1\%).

A bengala inteligente é voltada para pessoas com deficiência visual que desejam mais independência e segurança em suas atividades diárias. O dispositivo foi projetado especificamente para pessoas com vários graus de deficiência visual, desde baixa visão até cegueira total.

O público-alvo também são cuidadores, familiares e profissionais que prestam assistência a pessoas com deficiência visual. Esses indivíduos desempenham um papel fundamental no suporte e no uso correto da bengala inteligente, ajudando a integrar a tecnologia na vida diária dos usuários.

Também é oportuno destacar instituições e organizações que lidam com o bem-estar e a inclusão de pessoas com deficiência visual. Estas organizações podem colaborar no desenvolvimento, teste e implementação da bengala inteligente para garantir que o dispositivo atenda às necessidades específicas da comunidade.

Por fim, a bengala inteligente é direcionada a desenvolvedores, pesquisadores e profissionais de tecnologia assistiva interessados em soluções inovadoras para promover acessibilidade e inclusão. Estas partes interessadas desempenham um papel importante no desenvolvimento e disseminação da tecnologia para um público mais vasto.

\subsection{DEMOGRAFIA}
O público-alvo da bengala inteligente inclui uma ampla gama de pessoas com deficiência visual que procuram soluções tecnológicas inovadoras para melhorar a sua qualidade de vida e liberdade de movimento. Este grupo inclui pessoas de todas as idades, desde jovens estudantes até idosos com dificuldade de locomoção diária devido à deficiência visual.
Além disso, o grupo demográfico pode incluir pessoas com vários graus de deficiência visual, desde cegueira total até visão parcial, e abranger uma vasta gama de necessidades e preferências de tecnologia de apoio. Ela foi projetada para acessar e se adaptar a uma variedade de condições visuais, fornecendo suporte e recursos personalizados para atender às necessidades específicas de cada usuário.


Ela também pode agradar aos pais e responsáveis de crianças com deficiência visual que procuram uma solução inovadora para promover a independência e segurança dos seus entes queridos. Estes indivíduos desempenham um papel crítico na implementação e aceitação de tecnologias de apoio e têm um impacto direto na procura e aceitação da bengala no mercado.
O grupo demográfico também pode incluir profissionais de saúde e terapeutas que trabalham com pessoas com deficiência visual para recomendar e integrar soluções tecnológicas eficazes nos seus programas de reabilitação e apoio. Esses especialistas desempenham um papel fundamental na divulgação e utilização de tecnologias assistivas na comunidade de saúde.

A demografia também inclui organizações e instituições que trabalham para garantir a inclusão e acessibilidade dos deficientes visuais. Estas organizações incluem escolas especializadas, centros de reabilitação, associações para deficientes visuais e ONGs que procuram promover soluções tecnológicas inovadoras aos seus membros e comunidades.
A bengala inteligente também pode ser adotada por órgãos governamentais e agências reguladoras que buscam implementar políticas de acessibilidade e inclusão para deficientes visuais. A unidade desempenha um papel importante na criação de um ambiente para a inovação e a utilização de tecnologias de apoio a um nível social mais amplo.

Outro grupo demográfico importante são os desenvolvedores de tecnologia, engenheiros e pesquisadores que trabalham para melhorar a acessibilidade e a inclusão por meio de soluções inovadoras. Estes especialistas desempenham um papel fundamental no desenvolvimento, melhoria e implantação de tecnologias de apoio, como bengalas inteligentes, e impulsionam a inovação em tecnologia acessível.

Além do público-alvo designado, a bengala inteligente também pode atrair investidores e empreendedores interessados em apoiar iniciativas de impacto social e tecnológico. O potencial de mercado e o impacto positivo na vida das pessoas com deficiência visual fazem da bengala inteligente uma oportunidade atraente para investimento e desenvolvimento de negócios focados na acessibilidade e inclusão.

Dito isto, a demografia da bengala inteligente é diversificada e inclui uma variedade de partes interessadas, desde deficientes visuais até profissionais médicos, organizações, agências governamentais, desenvolvedores de tecnologia e investidores. O interesse e a aceitação das bengalas inteligentes são motivados pela necessidade de soluções inovadoras e acessíveis para melhorar a qualidade de vida das pessoas com deficiência visual e incluí-las nos diversos setores da sociedade.



