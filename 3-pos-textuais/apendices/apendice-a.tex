\apendice{POSTAGENS NO BLOG}
\label{ap:A}

\section{Semana 1 (06/03/2024) - Apresentação da disciplina }
Na semana passada, no dia 28/02, tivemos nossa primeira aula de PI1A5 com o Professor Johnata. Embora ainda não tenhamos feito decisões específicas sobre o projeto, o professor nos apresentou a disciplina e explicou o que será esperado de nós nas próximas aulas, além de detalhar as próximas entregas e como deverão ser feitas. 

Com base nessas informações, começamos a considerar possíveis temas para o projeto e iniciamos pesquisas preliminares.

\section{Semana 2 (13/03/2024) - Formação dos grupos }

 Após a apresentação inicial da disciplina, nós formamos o nosso grupo com seis integrantes e decidimos o nosso nome Patent Pending.

A ideia surgiu, ironicamente, por estarmos sem ideia de qual nome representaria perfeitamente a equipe e o tema que ainda não havia sido definido. Logo, optamos por uma patente pendente.

Com a definição de quem seriam os integrantes, foi definido que a nossa ferramenta principal de organização seria o Jira. Dessa forma, a separação de tarefas e visualização de prazos fica mais facilitada por ser uma ferramenta voltada ao gerenciamento de projetos.

Alguns tópicos apresentados na primeira aula já foram convertidos em tarefas pendentes, tais como criação deste blog, novo canal no youtube e pesquisa de temas possíveis.

\section{Semana 2 (20/03/2024) -  Discussão sobre os temas }
Nessa última semana, nós começamos a discutir os possíveis projetos que gostaríamos de desenvolver ao longo da disciplina nesses dois últimos semestres.

Entre dez temas iniciais, sugeridos pelos integrantes do grupo, nós escolhemos as cinco melhores ideias para discutir com o professor a respeito da viabilidade.

As nossas sugestões incluíram uma plataforma para contratação de artistas, comunidade virtual para leitores, ferramenta para gerenciamento do processo de estágio entre aluno e universidade, aplicação para controle financeiro e, por fim, uma bengala inteligente.

Após a discussão com o professor, optamos por desenvolver a bengala inteligente. As escolhas individuais não formaram um consenso, portanto, foi necessário argumentar os prós e contras de cada tema e partir da decisão da maioria.

A escolha final foi feita, sobretudo, pela relevância do tema e inovação do projeto. A bengala é voltada para pessoas com deficiência visual e, atualmente, não existem ferramentas acessíveis no mercado voltada a essa população. Após realizarmos uma pesquisa de mercado, notamos que não existem outras tecnologias assistivas focadas em manter um custo baixo e ir além de uma bengala comum.


\section{Semana 4 (27/03/2024)  - Apresentação do tema }

Iniciamos as pesquisas iniciais sobre o tema, argumentando sobre a viabilidade do projeto, relevância do tema, pesquisas sobre o assunto e propostas de funcionalidades.

Ao montar a apresentação para introduzir o nosso projeto à sala, utilizamos o LaTeX pela primeira vez com o intuito de estruturar os slides. Apesar da dificuldade inicial, utilizar o modelo pronto fornecido pelo professor facilitou bastante esse primeiro contato. Além disso, o fato do Overleaf proporcionar ferramentas que permitem facilmente assimilar o conteúdo textual com a parte correspondente no PDF auxiliou a compreender o que cada tag fazia.

Uma dificuldade deu-se pelo fato de que a compilação no Overleaf era constantemente interrompida por conta do limite de tempo estabelecido na versão gratuita. Após algumas tentativas de recompilação, dava certo. Ainda assim, foi um pequeno empecilho durante o desenvolvimento dos slides.

Nós apresentamos o projeto para a turma e, em seguida, recebemos feedback do professor acerca da apresentação. Um ponto positivo foi a evidente pesquisa que fizemos para poder argumentar e explicar sobre o tema, contudo, foi recomendado que não nos estendêssemos tanto para não tangenciar demais o assunto e acabar passando do tempo limite.

Para a próxima entrega, devemos também voltar a pesquisa para trazer outros trabalhos acadêmicos que tenham desenvolvido o mesmo projeto. A ideia é que possamos usar como referência e também apontar diferenciais da nossa sugestão.



\section{Semana 5 (03/04/2024)  - Refinamento da ideia }
Algumas decisões importantes foram feitas acerca da bengala, quais seriam as principais funcionalidades e os requisitos. 

Dessa forma, a equipe trabalhou em desenvolver as histórias de usuário do projeto e seus respectivos critérios de aceitação. Com esses pontos prontos, ficou mais fácil definir quais seriam os requisitos funcionais e não funcionais que atendessem a essas funcionalidades desejáveis. 

O desenho técnico do projeto foi iniciado, bem como o estudo do possível design da bengala considerando a maneira que pessoas com deficiência visual já utilizam bengalas convencionais. O propósito é que a bengala inteligente seja um substituto viável a alternativas analógicas comuns, portanto, foi importante pensar numa forma de preservar a mesma leveza e ergonomia.

A documentação do projeto também foi iniciada, trazendo a introdução do projeto e estudo das alternativas existentes no mercado. 

Houve discussões acerca da viabilidade das nossas propostas iniciais, sendo elas a integração com um aplicativo o principal desafio por questão de tempo e falta de familiaridade por parte do grupo com essa área de desenvolvimento. A possibilidade de não ter um aplicativo agora apresentou empecilhos para outras funcionalidades anteriormente idealizadas, como localização da bengala, reconhecimento de imagem e integração com um mapa para cálculo de rotas. 

Outra alternativa para manter algumas funcionalidades rodando localmente seria o uso de um microcontrolador mais potente, como Raspberry PI em vez de Arduino, uma vez que este último possui limitações de processamento, memória e linguagem de programação. Contudo, a escolha de um outro microcontrolador resulta no aumento de custos do projeto, nos levando a escolher o foco em outras funcionalidades por enquanto.


\section{Semana 6 (10/04/2024) - Desenho da aplicação }
 Na última semana, tivemos a oportunidade de discutir com o professor sobre as possíveis funcionalidades da bengala. Como o nosso objetivo é manter os custos baixos para tornar a ferramenta acessível, utilizar microcontroladores com maior poder de processamento e mais memória não é viável por enquanto.

No sábado, o grupo se reuniu para definir qual seria a segunda funcionalidade principal, além de definir melhor os responsáveis por cada segmento do projeto. Apesar de concordarmos com uma flexibilidade na designação de tarefas, foi importante decidir responsáveis por cada parte do trabalho para que fosse possível evitar a centralização de diversas atividades em poucas pessoas. Além disso, também permite uma gestão melhor, já que existem pessoas específicas pensando em todas as frentes de desenvolvimento.

Na reunião, alinhamos que a funcionalidade secundária escolhida seria a detecção do toque do usuário na bengala, de forma que ela emitisse alarme sonoro/vibratório após um longo período de inatividade na região do apoio de mão. Isso permite que o usuário possa desligar a bengala, se necessário, ou localizá-la facilmente no ambiente caso tenha perdido ou esquecido de pegá-la.

Foram feitos testes de componentes no Tinkercad, com a demonstração da vibração da bengala de acordo com a distância do objeto identificado. O uso de dois sensores ultrassônicos foi testado numa simulação a parte. Além disso, a modelagem 3D também foi realizada no mesmo software, escolhido pela possibilidade de montar circuitos virtualmente e uma ampla disponibilidade de componentes e peças. 

Contudo, apesar da amplitude de escolhas, o Tinkercad não oferece todos os componentes que escolhemos utilizar no nosso projeto. Assim, os componentes foram comprados na semana passada e possuem previsão de chegar nessa semana. Com isso, será possível testar a comunicação de mais sensores ao mesmo tempo, até que todos sejam integrados num único circuito no projeto finalizado.

A documentação foi finalizada também na última semana, bem como a apresentação em slides. Durante a apresentação do desenho da aplicação, levamos em consideração os pontos de melhora sugeridos pelo professor, como tempo de fala, quantidade de texto em slides e excesso de detalhes acerca de temas fora do escopo. Com isso, notou-se uma melhora significativa na qualidade geral da apresentação.

Assim que os componentes chegarem, será possível testá-los para que possíveis problemas nas peças sejam identificados e corrigidos a tempo da próxima apresentação.

\section{Semana 7 (17/04/2024) - Definição de teste dos componentes }
Na última aula, o professor especificou o que seria esperado da prova de conceito e dos testes de componentes. Assim, definimos quais aspectos dos componentes nós testaríamos e quais seriam os parâmetros desses testes.

Vamos utilizar algumas planilhas disponibilizadas para especificar cada um dos testes. Os que nós decidimos, até então, são:

1.  Sensor ultrassônico

    a. Distância mínima e máxima de detecção

    b. Variação da distância detectada pelo código e pela mensuração

    c. Ângulo de detecção

2. Motor de vibração

    a. Vibração mínima e máxima

3. Buzzer

    a. Decibéis do som emitido

4.  Sensor de toque

    a. Tempo máximo de toque contínuo para continuar detectando

    b. Tempo mínimo de toque para o reconhecimento

    c. Área de detecção do toque no sensor

    d. Outros objetos que ativam o toque

5. Bateria

    a. Possibilidade de energizar o microcontrolador com todos os componentes ao mesmo tempo

    b. Duração da bateria durante o uso contínuo

6.  Módulo de carregamento

    a. Carregamento da bateria sem esquentar

    b. Tempo de carregamento de acordo com a capacidade da bateria


Um obstáculo, até então, reside na dificuldade para medir com exatidão alguns componentes, a fim de deixar o teste mais preciso. O motor de vibração, por exemplo, precisa de um parâmetro para indicar que a sua resposta tátil é forte o suficiente quando estiver acoplada no corpo de uma bengala. Além disso, medir o nível de carregamento da bateria é difícil e pode não ser tão específico sem uma ferramenta específica.

\section{Semana 8 (24/04/2024) - Integração dos componentes }
Os componentes chegaram e foi possível testá-los individualmente e em conjunto. Descobrimos que o buzzer ativo possui um som muito alto e, por não permitir a alteração de frequência, pode ser um pouco incômodo durante a utilização. 

A bateria está funcionando normalmente e foi possível energizar todos os componentes ao mesmo tempo. Além disso, o módulo de carregamento indica o carregamento da bateria através de uma luz de LED acesa. 

O motor de vibração vibra bem, variando de 0 a 255 em intensidade, mas os níveis mais baixos podem ser imperceptíveis ao usuário quando o sensor for acoplado na bengala. Assim, precisamos realizar mais testes para determinar onde o motor deve ficar para que a vibração seja amplificada ou decidir se outros motores em conjunto serão necessários.

Os sensores ultrassônicos identificam objetos até uma boa distância, então o propósito será atendido. Apenas precisamos nos certificar de que ele é suficientemente preciso e consegue identificar objetos de diferentes alturas e tamanhos.

\section{Semana 9 (01/05/2024) - POC }
Iremos apresentar a prova de conceito na próxima semana. Assim, terminamos a apresentação de slides e definimos o escopo do que será apresentado à turma. 

Realizaremos a apresentação do funcionamento real dos transdutores (motor de vibração e buzzer) e sensores (capacitivo e ultrassônico). Assim, vamos criar um código específico para essa etapa a fim de acoplar todos os componentes:

1. Quando o sensor capacitivo identificar um toque, o buzzer ativará por alguns instantes.

    a. Demonstração da captação do toque e da resposta sonora

2. Dependendo da distância do objeto identificado pelo sensor ultrassônico, o motor vibrará mais ou menos.

    a. Demonstração da variação de vibração e da amplitude da distância detectada
    
\section{Semana 10 (08/05/2024) - Apresentação da POC }
 A prova de conceito foi apresentada para a turma hoje. A apresentação incluiu a introdução de todos os componentes integrados num circuito, de forma que a funcionalidade de todos pudesse ser testada e representada.

Assim, o código foi desenvolvido a fim de obter respostas de todos os sensores e, concomitantemente, estimular os transdutores.



\section{Semana 11 (17/07/2024) - Pós greve e finalização do documento}
Durante a terceira semana de maio, o professor da disciplina entrou em greve, a qual encerrou-se no dia 28 de junho. Assim, nossas aulas retornaram como período de reposição após o período de férias (01/07 até 15/07). 

Tivemos a nossa primeira aula após esse período hoje, na qual discutimos os futuros prazos de entrega e apresentações para finalização do quinto semestre. Dessa forma, o documento deverá ser entregue até o dia 23/07 com todos as informações até o MVP.

Em seguida, no dia 24/07, haverá a preparação para a apresentação que está prevista de ocorrer no dia 31/07. Com o trabalho apresentado para a banca, as alterações serão enviadas até o dia 07/08.

Recebemos feedback do professor quanto à documentação existente, com as seguintes sugestões de alteração antes da entrega:

- Incluir textos antes e depois de cada tabela, quadro, imagem ou lista dentro do documento

- Referenciar os itens acima no texto, garantindo que eles estejam associados com o que está escrito

- Alterar as tabelas para quadros

- Incluir referências no texto dos dados trazidos durante a argumentação



Junto com os pontos mencionados acima, será adicionada ao documento a seção de apresentação do MVP.
