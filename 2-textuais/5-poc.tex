\chapter{Prova de Conceito}
\label{chap:prova-de-conceito}

A Prova de Conceito, ou Proof of Concept (POC), é um estágio crucial no desenvolvimento de qualquer projeto tecnológico. Ela consiste em uma fase dedicada a demonstrar a viabilidade e eficácia de uma ideia ou conceito inicial, fornecendo evidências tangíveis de que o projeto pode ser implementado com sucesso. Em essência, uma Prova de Conceito é uma espécie de teste preliminar, no qual protótipos ou modelos simplificados são criados e avaliados. Seu objetivo principal é validar a funcionalidade principal do conceito proposto, identificando potenciais desafios técnicos, explorando soluções alternativas e determinando a viabilidade do projeto em um contexto real.

\section{Implementação da Prova de Conceito}

Para realizar a Prova de Conceito da Bengala Inteligente, foi desenvolvido um protótipo inicial que integra todos os principais componentes do sistema. Isso incluiu a montagem física da bengala com os sensores e transdutores necessários, bem como a programação do microcontrolador para controlar o comportamento do dispositivo em resposta aos dados dos sensores.

\subsection{Montagem do Protótipo}

A montagem do protótipo envolveu a conexão dos sensores capacitivo de toque e ultrassônico no microcontrolador, de forma que eles pudessem captar dados de toque humano e da localização de objetos em relação ao sensor, garantindo uma demonstração prática para a detecção de obstáculos e interações táteis. 

Além disso, foi integrado um módulo de \textit{buzzer} para fornecer alertas sonoros ao usuário, bem como o motor de vibração para a indicação tátil ao usuário.

O circuito foi energizado diretamente pela bateria, ambos conectados entre si através do módulo de gerenciamento de energia. Com isso, tornou-se possível demonstrar a funcionalidade de recarregar através de uma fonte externa, bem como evitar sobrecargas durante a demonstração.

Por fim, a chave switch demonstrou a interrupção da corrente sendo conectada a um LED, de forma que possibilitou o controle mecânico acerca da luz estar acesa ou não. 


\subsection{Programação do Microcontrolador}

Utilizando um microcontrolador Arduino, foi desenvolvida a lógica de controle para processar os dados dos sensores e acionar os transdutores conforme necessário. Isso envolveu a configuração dos pinos de entrada e saída, a leitura dos dados dos sensores e a implementação de algoritmos para tomada de decisão com base nessas informações.

Os dados captados pelo sensor capacitivo são utilizados como critério para acionar o \textit{buzzer}, proporcionando uma resposta audível quando ocorre um toque. Essa escolha foi deliberada em consonância com o propósito central da bengala no que se refere à funcionalidade do sensor de toque. Embora o requisito primordial envolva a geração de alertas sonoros quando não houver toque por um período definido, indicando situações como queda da bengala, esquecimento de ligá-la ou perda, durante a demonstração seria impraticável manter o sensor constantemente ativado para evitar sons indesejados que poderiam interferir na apresentação. Portanto, optou-se por uma abordagem inversa, ou seja, ativar o \textit{buzzer} enquanto o toque estiver sendo detectado. Além disso, foi implementado um LED para que permanecesse ligado juntamente ao \textit{buzzer}.

Por outro lado, a escala de demonstração do sensor ultrassônico foi limitada para o raio de até 30 centímetros. Dessa forma, tornou-se possível demonstrar a variação do motor de vibração de forma mais evidente.

\section{Testes e Validação}

Após a montagem e programação do protótipo, foram realizados testes para validar a funcionalidade e eficácia do sistema. Isso incluiu testes de detecção de obstáculos usando os sensores de ultrassom, testes de sensibilidade do sensor capacitivo \textit{touch} e testes de resposta do \textit{buzzer}.

\subsection{Sensor ultrassônico}
O sensor ultrassônico emite ondas ultrassônicas de alta frequência através do pin Trigger, que são refletidas quando atingem um objeto e identificadas pelo sensor. Assim, o dado de recebimento da onda é enviado ao microcontrolador através do pino Echo. O cálculo de tempo e distância é feito a nível do software, considerando uma aproximação da velocidade do som no ar.

Notou-se, de maneira empírica, que o sensor tem uma dificuldade para identificar com precisão objetos pequenos. Entretanto, obteve-se um resultado satisfatório no quesito precisão em objetos maiores e velocidade da resposta.

\subsection{Buzzer Ativo}
O Buzzer foi utilizado para emitir sons sem diferenciação de frequência assim que o sensor capacitivo identificasse um toque. Utilizando um aplicativo de celular com o propósito de um decibelímetro, mensurou-se que a pressão sonora dos sons emitidos por este transdutor fica em torno de 60 dB, faixa aproximada de uma conversa normal. Apesar de ser alto, ainda permanece confortável ao ouvido humano.

\subsection{Motor de vibração}
O motor escolhido realiza até 9000 rotações por minuto (RPM), o que gera a sensação de vibração. A sua intensidade pode variar de 0 a 255, de acordo com o valor atribuído pelo toque. Embora a vibração mais forte seja bastante perceptível, ela fica bastante sutil conforme o número de rotações diminui. Todos os modos são perceptíveis pelo contato direto, contudo, a ideia é que o módulo seja integrado no corpo da bengala, então existe o risco de não haver uma distinção clara entre os alertas para objetos mais distantes. Por esse motivo, a prova de conceito revelou que é necessário explorar diferentes formas de reposicionamento do motor no corpo da bengala ou reavaliar opções que vibrem mais intensamente.

\subsection{Sensor capacitivo \textit{touch}}

O sensor capacitivo \textit{touch} é um dispositivo eletrônico que detecta a presença de um objeto condutor, como um dedo humano, sem a necessidade de contato físico direto. Ele opera com base na capacidade de detectar mudanças no campo elétrico ao redor do sensor quando um objeto se aproxima. Observou-se que ele possui uma resposta rápida ao toque, entretanto, ele detectou toques com uma distância de aproximadamente 3 milímetros (mm) do objeto condutor real (nesse caso, o dedo). Portanto, a detecção não acontece somente quando há um toque direto.

Além disso, ao ser tocado, o sensor capacitivo \textit{touch} mantém seu estado de saída alto por um período de aproximadamente 12 segundos (s). Dessa forma, mesmo que o objeto condutor permaneça em contato, o sensor deixará de identificar o toque após 12 segundos. Assim, volta a identificar com uma leve movimentação. 

Embora não seja completamente impreciso, apresenta características que se tornam aceitáveis, dado o contexto e os requisitos do projeto.






\subsection{Chave DIP Switch}
A chave DIP (Dual In-line Package) Switch é um componente eletrônico utilizado em circuitos integrados e placas de circuito impresso para configurar diferentes estados ou opções de funcionamento. Consiste em uma série de interruptores dispostos em uma linha, onde cada interruptor pode ser posicionado para estar aberto (desligado) ou fechado (ligado), permitindo configurar um conjunto de valores binários.

Apesar de ter cumprido com o objetivo proposto, a chave é bastante pequena e a sua movimentação é rígida, dificultando o manuseio. Com isso, tornou-se importante avaliar outras opções pensando na utilização por parte do usuário final.







\section{Resultados e Considerações}

Os resultados da Prova de Conceito demonstraram a viabilidade técnica da Bengala Inteligente, com uma integração eficaz dos componentes e uma resposta satisfatória aos estímulos externos. No entanto, foram identificados alguns pontos que requerem ajustes e melhorias, como a interação com componentes e a resposta de transdutores.

No geral, a Prova de Conceito foi bem-sucedida em validar a funcionalidade principal da Bengala Inteligente e fornecer percepções acerca dos componentes utilizados.


\chapter{PRODUTO MÍNIMO VIÁVEL}
O Minimum Viable Product (MVP), ou Produto Mínimo Viável, representa uma etapa crucial no ciclo de desenvolvimento de qualquer projeto tecnológico. É uma versão inicial e simplificada do produto final, projetada para validar as funcionalidades essenciais e coletar feedback inicial dos usuários. Ao contrário de uma versão completa, o MVP foca em oferecer apenas as funcionalidades básicas necessárias para resolver o problema principal do usuário. Este estágio permite não apenas testar a aceitação do produto, mas também validar o conceito, identificar ajustes necessários e refinar a estratégia de desenvolvimento.

\section{Implementação do Produto Mínimo Viável}
Para implementar o MVP da Bengala Inteligente, foi desenvolvido um protótipo inicial que se concentra nas funcionalidades fundamentais essenciais para melhorar a mobilidade e segurança dos usuários com deficiência visual. Dessa forma, decidiu-se que o projeto incluiria detecção de obstáculos, vibração ajustável, circuito desligável e com possibilidade de recarregar a bateria. Com isso, a bengala torna-se funcional ainda no estágio inicial de desenvolvimento.

\section{Prototipação da Bengala}
O modelo virtual da bengala foi prototipado considerando uma estrutura para alocação dos componentes, passagem de fios, leveza e ergonomia.

O protótipo foi realizado no Autodesk Inventor, software que permite criar protótipos virtuais tridimensionais e funcionais. Como partes integrantes da bengala, há a ponteira, um compartimento para alocação dos componentes eletrônicos e, por fim, a extensão.

A bengala mede 125,42cm e possui 1,6cm de diâmetro, permitindo um fácil manuseio da bengala por usuários de diferentes alturas. A espessura aumenta próximo à mão, medindo 2,0cm, o que permite maior conforto.

O Apêndice \ref{ap:C} ilustra diferentes ângulos do modelo tridimensional da bengala, detalhando as dimensões de cada componente.

\section{Componentes elétricos}
Os dois sensores ultrassônicos foram conectados ao microcontrolador, em conjunto com o módulo de motor de vibração. Além disso, o microcontrolador é alimentado pela bateria através do módulo de gerenciamento de carga (CN3065), com a chave DIP Switch interceptando a passagem de energia para permitir o desligamento do circuito. 

Foi desenvolvida uma função que monitora a atividade do sensor, convertendo o tempo em microssegundos para a distância entre o sensor e o objeto detectado. Assim, essa distância é utilizada para definir a intensidade de vibração do motor.

As condições definidas para o controle da vibração são baseadas na distância medida pelos sensores. Quando a distância detectada por qualquer um dos sensores é menor que 50 cm, o motor é acionado com uma intensidade máxima de vibração (255). Se a distância for menor que 100 cm, mas não tão próxima quanto 50 cm, a intensidade da vibração é reduzida para 200. Da mesma forma, distâncias inferiores a 150 cm acionam uma vibração moderada de intensidade 150. Se nenhuma das condições anteriores for atendida, o motor vibrará continuamente com uma intensidade mínima de 100.

Essas definições permitem ajustar a intensidade da vibração de acordo com a proximidade de obstáculos detectados, proporcionando feedback tátil adequado ao usuário para navegação segura e eficiente.

Além disso, é importante garantir que os atrasos (delays) inseridos após cada alteração na intensidade da vibração permitam que o feedback seja percebido de forma clara e oportuna pelo usuário, contribuindo para uma experiência de uso mais intuitiva e eficaz da bengala inteligente.


\section{Integração dos Componentes}
Na etapa do MVP, os componentes ainda não estão soldados num circuito integrado, uma vez que o projeto pode sofrer alterações que envolvem a inclusão, exclusão ou reestruturação das peças no circuito. Pensando nisso, os componentes estão integrados na parte externa da bengala com fita adesiva, ainda com a utilização da protoboard para auxiliar na montagem do circuito.

\section{Justificativa}
A escolha das funcionalidades presentes no MVP foi feita baseando-se no intuito primordial da bengala, bem como em aspectos não-funcionais que garantiriam uma boa experiência ao usuário durante a utilização desde o início.

 O Quadro abaixo revela os requisitos funcionais e não funcionais abrangidos na bengala. Assim, de 21 requisitos levantados inicialmente, 14 estão sendo implementados ainda na primeira versão (isto é, cerca de 66\%).
 
\begin{quadro}[!ht]    
    \captionsetup{width=1.0\textwidth} % Definindo a largura da legenda
    \caption{Requisitos gerais do MVP}  
    \renewcommand{\arraystretch}{1.5} % Aumenta o espaçamento entre as linhas
    \begin{tabular}{p{0.4\textwidth}p{0.54\textwidth}} % Definindo larguras para as colunas
        \toprule
        Referência & Nome  \\
        RF01 & Identificação de objetos   \\
        RF02 & Altura de objetos  \\
        RF03 & Detecção em ambientes escuros    \\
        RF04 & Modo de vibração   \\
        RF05 & Diferença entre os modos de vibração \\
        RF06 & Recarregamento da bengala   \\
        RF07 & Indicador de carga completa   \\
        RF08 & Desligamento  \\
        RNF01 & Tempo de vibração  \\
        RNF02 & Tempo de detecção \\
        RNF03 & Precisão   \\
        RNF05 & Entrada do carregador  \\
        RNF06 & Proteção contra sobrecarga   \\
        RNF07 & Autonomia de bateria  \\
        \bottomrule
    \end{tabular}
    \caption*{Fonte: elaborada pelos autores.} % Legenda sem rótulo
\end{quadro}

\section{Resultados e Considerações}

Após a implementação do MVP, foram obtidos resultados significativos que validam a funcionalidade e viabilidade do projeto inicial. O protótipo desenvolvido concentrou-se nas funcionalidades essenciais para melhorar a mobilidade e segurança dos usuários com deficiência visual. Isso incluiu a detecção de obstáculos, vibração ajustável, circuito desligável e recarregável, permitindo que a bengala se tornasse funcional desde o estágio inicial de desenvolvimento.

Os sensores ultrassônicos integrados ao microcontrolador demonstraram precisão satisfatória na detecção de objetos sólidos e na diferenciação de distâncias conforme especificado nos requisitos funcionais. O feedback tátil através da vibração foi implementado com sucesso, proporcionando uma comunicação clara das informações ambientais ao usuário durante a navegação.

Além disso, o módulo de gerenciamento de carga (CN3065) mostrou-se eficaz na recarga da bateria da bengala, garantindo uma autonomia adequada para o uso diário. Esse aspecto é crucial para assegurar que o dispositivo seja prático e confiável para os usuários em suas atividades cotidianas.

Em termos de validação do conceito, o MVP foi bem-sucedido em demonstrar que as funcionalidades propostas podem resolver efetivamente os desafios enfrentados pelos usuários com deficiência visual. 